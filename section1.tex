\section{L'arminianisme}

Les \og cinq points de l'arminianisme \fg{} étaient les suivants~:
\suggest{Tu as cité les \og cinq points de l'arminianisme \fg dans l'intro,
 et Chuck Smith utilise le terme ici, pourquoi ne pas l'utiliser?}

\begin{enumerate}
  \item  \textsc{Le libre arbitre}
 \suggest{Le texte original ne parle que du libre arbitre}

Arminius croyait que la chute de l'homme
  \fixme{Homme avec une majuscule est un terme scientifique désignant l'espèce humaine}
  n'était pas totale,
 soutenant qu'il restait toujours assez de bonté en l'homme
 pour qu'il désire le salut en Jésus-Christ.

  \item  \textsc{L'élection conditionnelle}
 \fixme{La prédestination n'est pas un point qui oppose calvinisme et arianisme, les calvinistes croient à la prédestination}

Arminius croyait que l'élection était basée sur la prescience
 \fixme{foreknowledge = prescience}
 de Dieu qui connaît à l'avance ceux qui vont croire.
 \fixme{Croire et « répondre à l'appel » peuvent être deux choses différentes}
 L' \og acte de foi \fg de l'homme était donc vu comme la \og condition \fg{} 
 de son élection à la vie éternelle, puisque Dieu savait déjà qu'il allait exercer
 son \og libre arbitre \fg{} en réponse à Jésus-Christ.
 \fixme{Le texte est très différent de la traduction}

  \item  \textsc{La rédemption universelle}

Arminius prétendait que la rédemption était basée sur le fait
 que Dieu aime tous les hommes, que Christ est mort pour tous,
 et que le Père ne désire pas que tous périssent.
 La mort du Christ a fourni à Dieu les fondements
 pour sauver tous les hommes, mais chacun doit exercer son propre
 \og libre arbitre \fg{} afin d'être sauvé.

  \item  \textsc{La grâce résistible}
  \fixme{On parle de grâce résistible dans l'arminianisme par opposition à la grâce irrésistible}

Arminius croyait que puisque Dieu voulait que tous les hommes soient sauvés,
 Il a envoyé le Saint-Esprit pour \og courtiser \fg{}
 \fixme{woo = courtiser, c'est un terme fort choisi pour faire le parallèle avec l'église = épouse de Christ}
 tous les hommes au Christ,
 mais comme l'homme possède un \og libre-arbitre \fg{} absolu, il est capable de résister
 à la volonté de Dieu pour sa vie.
 Il croyait que la volonté de Dieu de sauver tous les hommes peut être
 entravée par la volonté finie de l'homme.
 Il enseignait également que l'homme exerce son propre libre arbitre,
 et n'est né de nouveau que par la suite.

  \item \textsc{Perte de la grâce}
  \fixme{Il s'agit de la perte de la grâce \ocadr du salut}

Si l'homme ne peut pas être sauvé par Dieu à moins de le vouloir 
 \fixme{Contre-sens~: il s'agit de la grâce irrésistible citée plus haut},
 alors l'homme ne pourra garder son salut qu'à la condition
 qu'il continue à souhaiter être sauvé.
 \fixme{Il continue à souhaiter, et non il souhaite continuer}

\end{enumerate}




