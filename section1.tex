\section{L'arminianisme}

Les \og cinq points de l'arminianisme \fg{} étaient les suivants~:
\suggest{Tu as cité les \og cinq points de l'arminianisme \fg dans l'intro,
 et Chuck Smith utilise le terme ici, pourquoi ne pas l'utiliser?}

\begin{enumerate}
  \item  \textsc{Le libre arbitre}
 \suggest{Le texte original ne parle que du libre arbitre}

Arminius croyait que la chute de l'Homme n'était pas totale,
 soutenant qu'il restait toujours quelque chose de bon en l'homme
 afin de lui permettre d'accepter Jésus-Christ en vue du salut.

  \item  \textsc{L'élection conditionnelle}
 \fixme{La prédestination n'est pas un point qui oppose calvinisme et arianisme, les calvinistes croient à la prédestination}

Arminius croyait que l'élection était basée sur l'omniscience de Dieu
 qui connaît à l'avance ceux qui vont répondre à l'appel.
 L'élection à la vie éternelle était donc déterminée ou conditionnée
 par ce que ferait l'homme. Il appartenait entièrement à l'homme
 de déterminer s'il était élu pour le salut.
 Dieu connaissait, et choisissait ceux qui,
 de leur propre volonté, allaient choisir le Christ.

  \item  \textsc{La rédemption universelle}

Arminius prétendait que l'œuvre rédemptrice du Christ rendait le salut possible
 à chacun, mais n'a véritablement assuré le salut de personne.
 Bien que le Christ soit mort pour tous les hommes,
 il n'y a que ceux qui croient en lui qui seront sauvés.
 La mort de Christ permettait à Dieu d'accorder le pardon aux pécheurs
 à la condition qu'ils croient en Lui, mais elle n'enlevait véritablement
 le péché de personne. La rédemption ne devenait effective
 que si l'homme choisissait de l'accepter.

  \item  \textsc{La grâce résistible}
  \fixme{On parle de grâce résistible dans l'arminianisme par opposition à la grâce irrésistible}

Arminius croyait que Dieu voulait que tous les hommes soient sauvés,
 et  qu'Il a donc envoyé le Saint-Esprit afin d'\og amener \fg{} tous les hommes au Christ,
 mais comme l'homme  a son \og libre-arbitre \fg{}, il est capable de résister
 à la volonté de Dieu pour sa vie. Ainsi, le libre-arbitre de l'homme
 limite l'Esprit dans l'application de l'œuvre du salut de Christ.
 Le Saint-Esprit ne peut attirer au Christ que ceux qui le laissent agir en eux.
 Tant que le pécheur n'a pas répondu, l'Esprit ne peut donner la vie.

  \item \textsc{Perte de la grâce}
  \fixme{Il s'agit de la perte de la grâce \ocadr du salut}

Si l'homme ne peut pas être sauvé par Dieu à moins de le vouloir 
 \fixme{Contre-sens~: il s'agit de la grâce irrésistible citée plus haut},
 alors l'homme ne pourra garder son salut qu'à la condition qu'il continue à souhaiter être sauvé.
 \fixme{Il continue à souhaiter, et non il souhaite continuer}

\end{enumerate}




