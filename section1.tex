%\chapter{L'arminianisme}

Les \Og cinq points de l'arminianisme \Fg{}
 étaient les suivants~:
\suggest{Tu as cité les \Og cinq points de l'arminianisme \Fg dans l'intro,
 et Chuck Smith utilise le terme ici, pourquoi ne pas l'utiliser?}

\section*{Le libre arbitre}

\begin{pocketpar}{\tolerance=6000}
Arminius croyait que la chute de
 l'homme n'était pas totale,
 soute\-nant qu'il restait toujours assez de bonté en l'homme
 pour qu'il désire le salut en Jésus-Christ.
\end{pocketpar}

\section*{L'élection conditionnelle}

\begin{pocketpar}{\tolerance=6000}
Arminius croyait que l'élection
 était basée sur la prescience
 de Dieu qui connaît à l'avance ceux qui vont croire.
 L'~\Og acte de foi \Fg{} de l'homme était donc vu comme la \Og condition \Fg{} 
 de son élection à la vie éternelle, puisque Dieu savait déjà
 que l'homme allait exercer son \Og libre arbitre \Fg{}
 en réponse à Jésus-Christ.
\end{pocketpar}

\section*{L'expiation illimitée}

Arminius soutenait que la rédemption était basée sur le fait
 que Dieu aime tous les hom\-mes, que Christ est mort pour tous,
 et que le Père ne veut pas qu'aucun périsse.
 \ibiblephantom{IIP}(3:9)
 La mort du Christ a fourni à Dieu les fondements
 pour sauver tous les
 hommes, mais chacun doit exercer son propre
 \Og libre arbitre \Fg{} afin
 d'être sauvé.

\section*{La grâce résistible}

\begin{pocketpar}{\tolerance=6000}
Arminius croyait que puisque Dieu voulait que tous les hommes soient sauvés,
 Il a envoyé le Saint-Esprit pour \Og courtiser \Fg{}
 tous les hommes au Christ,
 mais comme l'homme
 possède un \Og libre-arbitre \Fg{} absolu, il est capable de résister
 à la volonté de Dieu pour sa vie.
 Il croyait que la volonté de Dieu de sauver tous les hommes peut être
 entravée par la volonté finie de l'homme.
 Il enseignait également que l'homme exerce son propre libre arbitre,
 et n'est né de nouveau que par la suite.
\end{pocketpar}

\section*{La perte de la grâce}

Si l'homme ne peut pas être sauvé par Dieu à moins de le vouloir,
 alors l'homme ne pourra garder son salut qu'à la condition
 qu'il continue à souhaiter être sauvé.


%\closechapter


