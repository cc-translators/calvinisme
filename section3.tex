%\chapter[L'opinion de Calvary Chapel]{L'opinion de\\ Calvary Chapel}

\begin{pocketpar}{\tolerance=3000}
Notre but n'est pas de prendre position sur ces questions
  ou de créer des \pocketlinebreak
 divisions dans le corps du Christ
 sur des interprétations humaines de ces vérités bibliques
 concernant notre salut.
 Nous désirons simplement exposer comment nous,
 églises Calvary Chapel, comprenons l'en\-sei\-gnement de la Bible
 concernant ces questions.
\end{pocketpar}
 
\section*{La dépravation}

Nous croyons que tous ont péché
 (\ibibleverse{Rm}(3:23)) et
 sont incapables par leurs propres forces humaines de gagner ou
 de mériter le salut (\ibibleverse{Tt}(3:5)).
 Nous croyons que le
 salaire du péché est la mort (\ibibleverse{Rm}(6:23)),
 et qu'en dehors de la grâce de Dieu, personne ne peut être sauvé
 (\ibibleverse{Ep}(2:8-9)).
 Nous cro\-yons que personne n'est juste, ou
 capable de faire le bien
 (\ibibleverse{Rm}(3:10-12)), et que sauf par
 conviction de péché et par régénération du Saint-Esprit, personne ne peut être sauvé
 (\ibibleverse{Jn}(1:12-13);
 \ibiblechvs{Jn}(16:8-11);
 \ibibleverse{IP}(1:23-25)).
 L'homme est clairement déchu et perdu dans le péché.
 \begingroup\widowpenalties 3 10000 10000 0
 \par\endgroup


\section*{L'élection}

\begin{digestpar}{\tolerance=1000}
Nous croyons que Dieu a choisi le croyant avant la fondation du mon\-de
 (\ibibleverse{Ep}(1:4-6)), que sur la base de Sa prescience
 Il a prédestiné
 le croyant afin de le transformer à l'image de Son fils
 (\ibibleverse{Rm}(8:29-30)).
 Nous croyons que Dieu offre le salut à toute personne qui invoquera Son nom.
 \ibibleverse{Rm}(10:13) dit~: \og Car quiconque invoquera le nom du Seigneur sera sauvé. \fg{}
 Nous croyons aussi que Dieu appelle à Lui tous ceux qui croient en Son fils,
 Jésus-Christ (\ibibleverse{ICo}(1:9)).
Néanmoins, la Bible nous enseigne aussi qu'une invitation (ou un appel)
 est lancée à chacun d'entre nous, mais que peu l'accepteront.
 Nous voyons cet équilibre tout au long de l'Écriture.
 \ibibleverse{Ap}(22:17) révèle~:
 \og [\dots{}] que celui qui veut, prenne de l'eau de la vie gratuitement ! \fg{}
 \ibibleverse{IP}(1:2) nous dit que nous sommes \og élus selon la prescience de Dieu le Père,
 par la sanctification de l'Esprit, pour l'obéissance et l'aspersion du sang de Jésus-Christ. \fg{}
 \ibibleverse{Mt}(22:14) dit~: \og Car il y a beaucoup d'appelés, mais peu d'élus. \fg{}
 Dieu opère clairement le choix, mais l'homme doit aussi accepter l'invitation de Dieu au salut.
\end{digestpar}


\section*{L'expiation}

\begin{digestpar}{\tolerance=1000}
Nous croyons que Jésus-Christ est mort en propitiation
 (une satisfaction de la juste colère de Dieu contre le péché)
 \og pour le monde entier \fg{} (\ibibleverse{IJn}(2:2) ; \ibiblechvs{IJn}(4:9-10))
 et qu'Il rachète et pardonne quiconque croit en la mort et en la résurrection
 de Jésus-Christ comme son unique espoir de salut du péché, 
 de la mort et de l'en\-fer (\ibibleverse{Ep}(1:7); \ibibleverse{IP}(1:18-19)).
 Nous croyons que la vie éternelle est un don de Dieu (\ibibleverse{Rm}(6:23))
 et que \og quiconque croit \fg{} en Jésus-Christ ne périt pas,
 mais a la vie éternelle (\ibibleverse{Jn}(3:16-18)).
 \ibibleverse{ITm}(4:10) dit~:
 \og Nous avons mis notre espérance dans le Dieu vivant, qui est le Sauveur
 de tous les hommes, surtout des croyants. \fg{}
 \ibibleverse{He}(2:9) déclare que Jésus
 \og a été fait pour un peu de temps inférieur aux anges [\dots{}], couronné
 de gloire et d'honneur [\dots{}] ; ainsi, par la grâce de Dieu, il a goûté
 la mort pour tous. \fg{}
 Le sacrifice expiatoire de Jésus-Christ était clairement suffisant pour sauver
 la race humaine toute entière.
\end{digestpar}


\section*{La grâce}

\begin{digestpar}{\tolerance=1000}
Nous croyons que la grâce de Dieu n'est pas le résultat de l'effort
 ou du mérite humains
 (\ibibleverse{Rm}(3:24-28); \ibiblechvs{Rm}(11:6)),
 mais qu'elle est la réponse
 de la miséricorde et de l'amour de Dieu à ceux qui croiront en Son Fils
 (\ibibleverse{Ep}(2:4-10)).
 La grâce nous donne \fixme{Contre-sens: si, elle nous donne beaucoup au contraire}
 ce que nous ne méritons pas, ni ne pouvons gagner par notre comportement
 (\ibibleverse{Rm}(11:6)).
 Nous croyons que nous pouvons résister à la grâce et à la miséricorde de Dieu.
 Jésus a dit dans \ibibleverse{Mt}(23:37)~:
 \og Jérusalem, Jérusalem, qui tues les prophètes et qui lapides ceux
 qui te sont envoyés, combien de fois ai-je voulu rassembler tes enfants,
 comme une poule rassemble ses poussins sous ses ailes, et vous ne l'avez pas voulu ! \fg{}
 Une personne n'est pas condamnée par manque d'opportunité d'être sauvée, 
 mais parce qu'elle fait le choix de ne pas croire
 (\ibibleverse{Jn}(3:18)).
 Dans \ibibleverse{Jn}(5:40), nous lisons~:
 \og Et vous ne voulez pas venir à moi pour avoir la vie ! \fg{}
 Jésus a également dit dans \ibibleverse{Jn}(6:37)~:
 \og Tout ce que le Père me donne viendra à moi,
 et je ne jetterai point dehors celui qui vient à moi. \fg{}
 \ibibleverse{Jn}(6:40) déclare~:
 \og Voici, en effet, la volonté de mon Père~: que quiconque voit le Fils
 et croit en lui ait la vie éternelle. \fg{}
 Dans \ibibleverse{Jn}(7:37), Jésus a dit~:
 \og Si quelqu'un a soif, qu'il vienne à moi et qu'il boive. \fg{}
 Dans \ibibleverse{Jn}(11:26), Il ajoute~:
 \og Quiconque vit et croit en moi ne mourra jamais. \fg{}
\end{digestpar}

Jésus reconnaît clairement la résistance et le rejet de l'hom\-me.
 Dans \ibibleverse{Jn}(12:46-48), Il a dit~:

\begin{quote}
 \og Moi, la lumière, je suis venu dans le monde, afin que quiconque croit
 en moi ne demeure pas dans les ténèbres. Si quelqu'un entend mes paroles
 et ne les garde pas, ce n'est pas moi qui le juge, car je suis venu non
 pour juger le monde, mais pour sauver le monde. Celui qui me rejette et qui
 ne reçoit pas mes paroles, a son juge~: la parole que j'ai prononcée,
 c'est elle qui le jugera au dernier jour. \fg{}
\end{quote}

\begin{digestpar}{\tolerance=1000}
Dans le message d'Étienne dans
 \ibibleverse{Ac}(7:51), il conclut en disant~:
 \og Hommes au cou raide, incirconcis de cœur et d'oreilles !
 vous vous opposez toujours au Saint-Esprit, vous comme vos pères. \fg{}
 Dans \ibibleverse{Rm}(10:21),
 l'apôtre Paul cite \ibibleverse{Is}(65:2)
 quand il parle des paroles de Dieu à Israël~:
 \og Tout le jour j'ai tendu mes mains vers un peuple rebelle et contredisant. \fg{}
 Dans un des cinq passages d'avertissement de l'Épître aux Hébreux,
 nous lisons dans \ibibleverse{He}(10:26)~:
 \og Car si nous péchons volontairement après avoir reçu la connaissance de la vérité,
 il ne reste plus de sacrifice pour les péchés. \fg{}
 Le verset \ibiblevs{He}(10:29) ajoute~:
\end{digestpar}

\begin{quote}
\begin{digestpar}{\tolerance=2000}
 \og Combien pire, ne pensez-vous pas, sera le châtiment mérité par celui qui
 aura foulé aux pieds le Fils de Dieu, tenu pour profane le sang de l'alliance
 par lequel il avait été sanctifié, et qui aura outragé l'Esprit de la grâce ! \fg{}
\end{digestpar}
\end{quote}

Il est clair que l'exercice du libre arbitre peut permettre aussi bien de résister
 à la grâce de Dieu que de la recevoir.

\section*{La persévérance}

\begin{digestpar}{\tolerance=2000}
Nous croyons que rien ne peut nous séparer de l'amour de Dieu en Jésus-Christ notre Seigneur
 (\ibibleverse{Rm}(8:38-39))
 et qu'il n'y a aucune condamnation pour ceux qui sont en Jésus-Christ
 (\ibibleverse{Rm}(8:1)).
 Nous croyons que la promesse de Jésus dans \ibibleverse{Jn}(10:27-28) est claire~:
 \og Mes brebis entendent ma voix. Moi, je les connais, et elles me suivent.
 Je leur donne la vie éternelle ; elles ne périront jamais,
 et personne ne les arrachera de ma main. \fg{}
 Jésus a dit dans \ibibleverse{Jn}(6:37)~:
 \og Tout ce que le Père me donne viendra à moi,
 et je ne jetterai point dehors celui qui vient à moi. \fg{}
 Nous avons l'assurance dans \ibibleverse{Ph}(1:6) \og que celui qui a commencé
 en [n]ous cette œuvre bonne, en poursuivra l'achèvement jusqu'au jour
 du Christ-Jésus. \fg{}
 Nous croyons que le Saint-Esprit nous a scellés
 pour le jour de la rédemption (\ibibleverse{Ep}(1:13-14);
 \ibiblechvs{Ep}(4:30)).
\end{digestpar}

Mais nous sommes également profondément préoccupés par les mots de Jésus dans \ibibleverse{Mt}(7:21-23)~:

\begin{quote}
\begin{digestpar}{\tolerance=3000}
 \og Quiconque me dit~: Seigneur,
 Seigneur ! n'entrera pas forcément
 dans le royaume des cieux, mais celui-là seul qui fait la volonté de mon Père
 qui est dans les cieux. Beaucoup me diront en ce jour-là~: Seigneur, Seigneur !
 N'est-ce pas en ton nom que nous avons prophétisé, en ton nom que nous avons
 chassé des démons, en ton nom que nous avons fait beaucoup de miracles ?
 Alors je leur déclarerai~: Je ne vous ai jamais connus retirez-vous de moi,
 vous qui commettez l'iniquité. \fg{}
\end{digestpar}
\end{quote}

Apparemment, beaucoup se proclament croyants mais en réalité ne le sont pas.

Jésus a dit dans \ibibleverse{Lc}(9:62)~: \og Quiconque met la main à la charrue,
 et regarde en arrière, n'est pas bon pour le royaume de Dieu. \fg{}
 \ibibleverse{ICo}(6:9-10) insiste sur le fait que
 \og les injustes n'hériteront pas le royaume de Dieu \fg{}
 et nous met en garde de ne pas être trompé.
 Suit une liste de styles de vie immoraux suivie d'une remarque
 indiquant qu'ils ne nous permettront pas d'hériter le royaume de Dieu.
 Des déclarations et des conclusions similaires nous sont données
 dans \ibibleverse{Ga}(5:19-21) et \ibibleverse{Ep}(5:3-5).

\ibibleverse{Ga}(5:4) dit~:
 \og Vous êtes séparés de Christ, vous qui cherchez la justification
 dans la loi ; vous êtes déchus de la grâce. \fg{}
 \ibibleverse{Col}(1:21-23) dit de Jésus-Christ~:
 \og Il vous a maintenant réconciliés par la mort dans le corps de sa chair,
 pour vous faire paraître devant lui saints, sans défaut et sans reproche ;
 si vraiment vous demeurez dans la foi, fondés et établis pour ne pas être
 emportés loin de l'espérance de l'Évangile que vous avez entendu,
 qui a été prêché à toute créature sous le ciel, et dont moi Paul
 je suis devenu le serviteur. \fg{}
 \ibibleverse{IITm}(2:12) dit que \og si nous le renions, Lui aussi nous reniera. \fg{}
 \ibibleverse{He}(3:12) dit~:
 \og Prenez donc garde, frères, que personne parmi vous n'ait un cœur méchant
 et incrédule, au point de se détourner du Dieu vivant. \fg{}
 Est-ce que de vrais croyants (\og frères \fg{}) peuvent se détourner du Dieu Vivant ?
 \ibibleverse{ITm}(4:1) nous apprend que \og dans les derniers temps,
 quelques-uns abandonneront la foi. \fg{}
 \ibibleverse{IITh}(2:3) parle de \og séduction \fg{} ou d'une apostasie.
 \ibibleverse{IIP}(2:20-21) constate remarquablement~:

\begin{quote}
 \og En effet, si après s'être retirés
 des souil\-lures du monde par la connaissance du Seigneur et Sauveur Jésus-Christ,
 ils s'y engagent de nouveau et sont vaincus par elles, leur dernière condition
 est pire que la première.
 Car mieux valait, pour eux, n'avoir pas connu la voie de la justice,
 que de l'avoir connue et de se détourner du saint commandement
 qui leur avait été donné. \fg{}
\end{quote}

\begin{digestpar}{\BRallowhypbch}
Il n'est pas étonnant que Pierre dise dans \ibibleverse{IIP}(1:10)~:
 \og C'est pourquoi frères, efforcez-vous d'autant plus d'affermir
 votre vocation et votre élection~: en le faisant, vous ne broncherez jamais. \fg{}
Nous remercions Dieu pour l'encouragement qu'Il nous donne dans \ibibleverse{Jude}(1:24)~:
 \og À celui qui peut vous préserver de toute chute et vous faire paraître
 devant sa gloire, irréprochables dans l'allégresse. \fg{}
\end{digestpar}

Maintenir un équilibre centré sur la Bible concernant ces questions difficiles est très important.
 Nous croyons vraiment dans la persévérance des saints (vrais croyants),
 mais nous nous sentons profondément concernés par les styles de vie pécheresse
 et les cœurs indociles parmi ceux qui s'appellent \og chrétiens \fg{}.
 Nous n'avons pas toutes les réponses à ces questions,
 mais nous désirons être fidèles au Seigneur et à Sa Parole.
 Lorsque nous nous retrouvons à fonder notre vision du salut
 sur le comportement et l'attitude des gens, cela nous
 décourage et nous préoccupe.
 Mais lorsque nous gardons nos yeux fixés sur le Seigneur et gardons confiance en Lui seul
 et en Sa puissance, nous disons alors avec Pierre dans
 \ibibleverse{IP}(1:3-9)~:

\begin{quote}
\begin{digestpar}{\tolerance=2000}
 \og Béni soit le Dieu et Père de notre Seigneur Jésus-Christ qui,
 selon sa grande miséricorde, nous a régénérés, par la résurrection
 de Jésus-Christ d'entre les morts, pour une espérance vivante,
 pour un héritage qui ne peut ni se corrompre, ni se souiller,
 ni se flétrir et qui vous est réservé dans les cieux,
 à vous qui êtes gardés en la puissance de Dieu, par la foi,
 pour le salut prêt à être révélé dans les derniers temps.
 Vous en tressaillez d'allégresse, quoique vous soyez maintenant,
 pour un peu de temps, puisqu'il le faut, affligés par diverses épreuves,
 afin que votre foi éprouvée \ocadr bien plus précieuse que l'or périssable,
 cependant éprouvé par le feu \fcadr{} se trouve être un sujet de louange,
 de gloire et d'honneur, lors de la révélation de Jésus-Christ.
 Vous l'aimez sans l'avoir vu. Sans le voir encore, vous croyez en lui
 et vous tressaillez d'une allégresse indicible et glorieuse,
 en remportant pour prix de votre foi le salut de vos âmes. \fg{}
\end{digestpar}
\end{quote}

\begin{pocketpar}{\tolerance=3000}
\begin{digestpar}{\tolerance=2000}
Il n'est pas facile de maintenir l'unité de l'Esprit parmi nous sur ces questions.
 La souveraineté de Dieu et la responsabilité humaine semblent être
 comme deux lignes parallèles qui ne se croiseront jamais dans nos esprits finis.
 Les voies de Dieu sont \og incompréhensibles \fg{} (\ibibleverse{Rm}(11:33))
 et la Bible nous met en garde de \og ne pas nous appuyer sur notre intelligence \fg{}
 (\ibibleverse{Pr}(3:5)).
 Dire ce que Dieu dit dans la Bible \ocadr ni plus, ni moins \fcadr{} n'est pas
 toujours facile, agréable, ou complètement compréhensible.
 Mais l'Écriture nous dit que la sagesse d'en haut sera aimante
 et douce envers tous ceux qui recherchent l'unité des croyants,
 n'essayant pas de diviser et de séparer les uns des autres.
 compatissants, nous faisant grâce réciproquement,
 comme Dieu nous a fait grâce en Christ. (\ibibleverse{Ep}(4:32))!
 Que dans les questions doctrinales difficiles, nous puissions nous comporter
 avec grâce et avec des cœurs remplis d'humilité, désirant avant tout
 plaire à Celui qui nous a appelés à Le servir dans le corps de Christ.
 Débat \ocadr \textsc{Oui} !
 Désaccords \ocadr \textsc{Oui} !
 Division \ocadr \textsc{Non} !
\end{digestpar}
\end{pocketpar}

Jésus a dit~: \og Vous les reconnaîtrez à leurs fruits. \fg{}
 \ibiblephantom{Mt}(7:16)Lorsqu'une position particulière
 sur les Écritures est source de conflit, de légalisme ou de division,
 je mets en doute la légitimité de cette position.
 Je cherche plutôt à tendre vers les choses qui vont me rendre plus aimant et bon,
 pardonnant plus et faisant preuve de plus de miséricorde\fixme{mercy = miséricorde}.
 Je sais alors que je deviens de plus en plus à l'image de mon Seigneur.
Si vous êtes arrivé à une intime conviction unilatérale sur question doctrinale,
 merci de nous accorder le privilège de constater d'abord comment cela
 vous a aidé à devenir plus semblable à Christ dans votre nature,
 et ensuite nous jugerons si nous avons besoin d'acquérir la même conviction.
 Assurons-nous toujours de regarder le fruit de l'enseignement.
 \fixme{Contre-sens}

\begin{pocketpar}{\tolerance=5000}
Recherchez les choses qui produisent la nature aimante de Jésus dans nos vies.
 \pocketlinebreak
 Je préfèrerais avoir une mauvaise connaissance mais une attitude juste,
 plutôt qu'une connaissance juste et une mauvaise attitude.
 Dieu peut changer ma compréhension des choses en un instant,
 mais cela prend bien souvent toute une vie pour que mon attitude
 soit transformée.
 \begingroup\widowpenalties 6 10000 10000 10000 10000 10000 0
 \par\endgroup
\end{pocketpar}


\nobreak
Affectueusement,

\nobreak
\signature{Chuck Smith}

\closechapter

