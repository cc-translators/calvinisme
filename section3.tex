\section{L'opinion de Calvary Chapel}

\mylettrine{N}{otre but} n'est pas de prendre position sur ces questions
  ou de créer des divisions dans le corps du Christ
 sur des interprétations humaines de ces vérités bibliques
 concernant notre salut.
 Nous désirons simplement exposer comment nous,
 églises Calvary Chapel, comprenons l'enseignement de la Bible
 quant à ces questions.
 
\begin{enumerate}

  \item  \textsc{La dépravation}

Nous croyons que tous, nous sommes pécheurs (\ibibleverse{Rm}(3:23))
 et incapables par nos propres forces humaines de gagner ou
 de mériter le salut (\ibibleverse{Tt}(3:5)).
 Nous croyons que le salaire du péché est la mort (\ibibleverse{Rm}(6:23)),
 et qu'en dehors de la grâce de Dieu, personne ne peut être sauvé
 (\ibibleverse{Ep}(2:8-9)).
 Nous croyons que personne n'est juste, ou capable de faire le bien
 (\ibibleverse{Rm}(3:10-12)), et que sauf par conviction,
 et par régénération du Saint-Esprit, personne ne peut être sauvé
 (\ibibleverse{Jn}(1:12-13)(16:8-11);\ibibleverse{IP}(1:23-25)).
 L'homme est clairement déchu et perdu dans le péché.


  \item  \textsc{L'élection}

Nous croyons que Dieu a choisi le croyant avant la fondation du monde
 (\ibibleverse{Ep}(1:4-6)), que sur la base de Sa prescience \fixme{prescience/omniscience}
 Il a prédestiné le croyant afin de le transformer à l'image de Son fils
 (\ibibleverse{Rm}(8:29-30)).
 Nous croyons que Dieu offre le salut à toute personne qui invoquera Son nom.
 \ibibleverse{Rm}(10:13) dit~: \og Car quiconque invoquera le nom du Seigneur sera sauvé. \fg{}
 Nous croyons aussi que Dieu appelle à Lui tous ceux qui croient en Son fils,
 Jésus-Christ (\ibibleverse{ICo}(1:9)).
Néanmoins, la Bible nous enseigne aussi qu'une invitation (ou un appel)
 est lancée à chacun d'entre nous, mais que peu l'accepteront.
 Nous voyons cet équilibre à travers de l'Écriture Sainte.
 \ibibleverse{Ap}(22:17) révèle~:
 \og [\dots{}] que  celui qui veut prenne de l'eau de vie, gratuitement. \fg{}
 \ibibleverse{IP}(1:2) nous dit que nous sommes \og élus selon la prescience de Dieu le Père,
 par la sanctification de l'Esprit, afin qu'ils deviennent obéissants et qu'ils participent
 à l'aspersion du sang de Jésus-Christ. \fg{}
 \ibibleverse{Mt}(22:14) dit~: \og Car il y a beaucoup d'appelés mais peu d'élus. \fg{}
 Dieu nous choisit clairement mais l'homme doit aussi accepter l'invitation de Dieu au salut.


  \item  \textsc{L'expiation}

Nous croyons que Jésus-Christ est mort en propitiation
 \fixme{propitiation et expiation sont deux choses différentes}
 (la satisfaction de la juste colère de Dieu contre le péché)
 \og pour le monde entier \fg{} (\ibibleverse{IJn}(2:2)(4:9-10))
 et qu'Il rachète et pardonne quiconque croit en la mort et en la résurrection
 de Jésus-Christ comme son unique espoir de salut du péché, 
 de la mort et de l'enfer (\ibibleverse{Ep}(1:7); \ibibleverse{IP}(1:18-19)).
 Nous croyons que la vie éternelle est un don de Dieu (\ibibleverse{Rm}(6:23))
 et que \og quiconque croit \fg{} en Jésus-Christ ne périt pas,
 mais a la vie éternelle (\ibibleverse{Jn}(3:16-18)).
 \ibibleverse{ITm}(4:10) dit~:
 \og Nous mettons notre espérance dans le Dieu vivant qui est le Sauveur de tous les hommes,
 surtout des croyants. \fg{}
 \ibibleverse{He}(2:9) déclare que Jésus \og a été abaissé pour un peu de temps au-dessous des anges,
 [\dots{}] couronné de gloire et d'honneur a cause de la mort qu'il a soufferte;
 ainsi par la grâce de Dieu, il a souffert la mort pour tous. \fg{}
 Le sacrifice expiatoire de Jésus-Christ était clairement suffisant pour sauver
 la race humaine toute entière.


  \item  \textsc{La grâce}

Nous croyons que la grâce de Dieu n'est pas le résultat de l'effort humain ou du mérite
 (\ibibleverse{Rm}(3:24-28)(11:6)), mais qu'elle est la réponse
 de la miséricorde et de l'amour de Dieu à ceux qui croiront en Son Fils
 (\ibibleverse{Ep}(2:4-10)).
 La grâce nous donne \fixme{Contre-sens: si, elle nous donne beaucoup au contraire}
 ce que nous ne méritons pas, et ne nous ne pouvons gagner par notre comportement
 (\ibibleverse{Rm}(11:6)).
 Nous croyons que nous pouvons résister à la grâce et la miséricorde de Dieu.
 Jésus a dit dans \ibibleverse{Mt}(23:37)~:
 \og Ô Jérusalem, Jérusalem qui tues les prophètes
 et qui lapides ceux qui te sont envoyés, combien de fois ai-je voulu rassembler tes enfants,
 comme une poule rassemble ses poussins sous ses ailes et vous ne l'avez pas voulu ! \fg{}
 Nous ne sommes pas condamnés parce que nous n'avons aucune opportunité d'être sauvé
 mais une personne est condamnée parce que qu'elle fait le choix de ne pas croire
 (\ibibleverse{Jn}(3:18)).
 Dans \ibibleverse{Jn}(5:40), nous lisons~:
 \og Et vous ne voulez pas venir à moi pour avoir la vie. \fg{}
 Jésus a également dit dans \ibibleverse{Jn}(6:37)~:
 \og Tous ceux que le Père me donne viendront à moi,
 et je ne mettrai pas dehors celui qui vient à moi. \fg{}
 \ibibleverse{Jn}(6:40) déclare~:
 \og La volonté de mon Père, c'est que quiconque voit le Fils
 et croit en lui ait la vie éternelle. \fg{}
 Dans \ibibleverse{Jn}(7:37), Jésus a dit~:
 \og Si quelqu'un a soif, qu'il vienne à moi, et qu'il boive. \fg{}
 Dans \ibibleverse{Jn}(11:26), Il ajoute~:
 \og Quiconque vit et croit en moi ne mourra jamais. \fg{}

Jésus reconnaît clairement la résistance et le rejet de l'homme.
 Dans \ibibleverse{Jn}(12:46-48), Il a dit~:
 \og Je suis venu comme une lumière dans le monde afin que quiconque croit en  moi
 ne demeure pas dans les ténèbres. Si quelqu'un entend mes paroles et ne les garde point ;
 ce n'est pas moi qui le juge, car je suis venu non pour juger le monde, mais pour sauver le monde.
 Celui qui me rejette et qui ne reçoit pas mes paroles a son juge ;
 la parole que j'ai annoncée, c'est elle qui le jugera au dernier jour. \fg{}

Dans le message d'Étienne dans \ibibleverse{Ac}(7:51), ses derniers mots furent~:
 \og Hommes au cou raide, incirconcis de cœur et d'oreille,
 vous vous opposez toujours au Saint-Esprit.
 Ce que vos pères ont été vous l'êtes aussi. \fg{}
 Dans \ibibleverse{Rm}(10:21), l'apôtre Paul cite \ibibleverse{Is}(65:2)
 quand il parle des paroles de Dieu à Israël~:
 \og J'ai tendu mes mains vers un peuple rebelle,
 qui marche dans une voie mauvaise au gré de ses pensées. \fg{}
 Dans un des cinq passages nous mettant en garde dans l'Épître aux Hébreux,
 \fixme{On ne dit pas vraiment Livre des Hébreux en français, ou alors il s'agit du TaNaK peut être}
 nous lisons dans \ibibleverse{He}(10:26)~:
 \og Car si nous péchons volontairement après avoir reçu la connaissance de la vérité,
 il ne reste plus de sacrifice pour les péchés. \fg{}
 Le verset \ibiblevs{He}(10:29) ajoute~: \og De quel pire châtiment pensez-vous
 que sera jugé digne celui aura foulé aux pieds le Fils de Dieu,
 qui aura tenu profane le sang de l'alliance, par lequel il a été sanctifié,
 et qui aura outragé l'Esprit de la grâce ? \fg{}
 Il est clair que l'exercice du libre arbitre peut aussi bien résister
 à la grâce de Dieu que la recevoir.

  \item  \textsc{La persévérance}

Nous croyons que rien ne peut nous séparer de l'amour de Dieu en Jésus-Christ notre Seigneur
 (\ibibleverse{Rm}(8:38-39)) et qu'il n'y a aucune condamnation pour ceux qui sont en Jésus-Christ
 (\ibibleverse{Rm}(8:1)).
 Nous croyons que la promesse de Jésus dans \ibibleverse{Jn}(10:27-28) est claire~:
 \og Mes brebis entendent ma voix ; je les connais et elles me suivent.
 Je leur donne la vie éternelle ; et elles ne périront jamais,
 et personne ne les ravira de Ma main. \fg{}
 Jésus a dit dans \ibibleverse{Jn}(6:37)~:
 \og Tous ceux que le Père me donne viendront à moi, et je ne mettrai pas dehors celui qui vient à Moi. \fg{}
 Nous avons l'assurance dans \ibibleverse{Ph}(1:6) \og que celui qui a commencé en vous
 cette bonne œuvre la rendra parfaite pour le jour de Jésus-Christ. \fg{}
 Nous croyons que le Saint-Esprit nous a scellés
 pour le jour de la rédemption (\ibibleverse{Ep}(1:13-14)(4:30)).

Mais nous sommes également profondément préoccupés par les mots de Jésus dans \ibibleverse{Mt}(7:21-23)~:
 \og Ceux qui me disent~: Seigneur, Seigneur ! N'entreront pas tous dans le royaume des cieux,
 mais seulement celui qui fait la volonté de mon Père qui est dans les cieux.
 Plusieurs me diront en ce jour-là~: Seigneur, Seigneur n'avons nous pas prophétisé par  ton Nom?
 N'avons-nous pas chassé des démons par ton nom? Alors je leur dirai ouvertement~:
 Je ne vous ai jamais connus, retirez-vous de moi je ne vous jamais connus, retirez-vous de moi,
 vous qui commettez l'iniquité. \fg{}
 Apparemment, beaucoup se proclament croyants mais en réalité ne le sont pas.

Jésus a dit dans \ibibleverse{Lc}(9:62)~: \og Quiconque met la main à la charrue,
 et regarde en arrière, n'est pas propre au royaume de Dieu. \fg{}
 \ibibleverse{ICo}(6:9-10) insiste sur le fait que
 \og les injustes n'hériteront pas du royaume de Dieu \fg{}
 et nous met en garde de ne pas être trompé. \fixme{Contre-sens: deceived = trompé}
 Suit une liste de styles de vie immoraux suivie d'une remarque
 indiquant qu'ils ne nous permettront pas d'hériter le royaume de Dieu.
 Des déclarations et des conclusions similaires nous sont données
 dans \ibibleverse{Ga}(5:19-21) et \ibibleverse{Ep}(5:3-5).

\ibibleverse{Ga}(5:4) dit~: \og Vous êtes séparés de Christ,
 vous tous qui cherchez la justification dans la loi ;
 vous êtes déchus de la grâce. \fg{}
 \ibibleverse{Col}(1:21-23) dit que Jésus-Christ
 \og [n]ous a réconciliés par sa mort dans le corps de sa chair,
 pour [n]ous faire paraître devant lui saints ; irrépréhensibles et sans reproche,
 si du moins [n]ous demeur[ons] fondés et inébranlables dans la foi,
 sans [n]ous détourner de l'espérance de l'Évangile que [n]ous av[ons] entendu,
 qui a été prêché à toute créature sous le ciel, et dont moi Paul, j'ai été fait ministre. \fg{}
 \ibibleverse{IITm}(2:12) dit que \og si nous le renions, lui aussi nous reniera. \og
 \ibibleverse{He}(3:12)~: \og Prenez garde, frères, que quelqu'un de vous n'ait un cœur mauvais
 et incrédule, au point de se détourner du Dieu vivant. \fg{}
 Est-ce que de vrais croyants (\og frères \fg{}) peuvent se détourner du Dieu Vivant ?
 \ibibleverse{ITm}(4:1) nous apprend que \og dans les derniers temps, quelques-uns abandonneront la foi. \fg{}
 \ibibleverse{IITm}(2:3) parle de \og séduction \fg{} ou d'une apostasie.
 \ibibleverse{IIP}(2:20-21) constate remarquablement~: \og En effet, si après s'être retirés
 des souillures du monde, par la connaissance du Seigneur et Sauveur Jésus-Christ,
 ils s'y engagent de nouveau et sont vaincus, leur dernière condition est pire que la première.
 Car mieux valait pour eux n'avoir pas connu la voie de la justice,
 que de l'avoir connue et de se détourner du saint commandement  qui leur avait été donné. \fg{}

Il n'est pas étonnant que Pierre dise dans \ibibleverse{IIP}(1:10)~:
 \og C'est pourquoi, frères, appliquez-vous d'autant plus à affermir
 votre vocation et votre élection;
 car, en faisant cela, vous ne broncherez jamais. \fg{}
Nous remercions Dieu pour l'encouragement qu'Il nous donne dans \ibibleverse{Jude}(1:24)~:
 \og (\dots{}) celui qui peut vous préserver de toute chute et vous faire paraître
 devant sa gloire irrépréhensibles et dans l'allégresse. \fg{}

Maintenir un équilibre centré sur la Bible concernant ces questions difficiles est très important.
 Nous croyons vraiment dans la persévérance des Saints (vrais croyants),
 mais nous nous sentons profondément concernés par les styles de vie pécheresse et des cœurs indociles
 de ceux qui s'appellent \og Chrétiens \fg{}.
 Nous n'avons pas toutes les réponses concernant ces questions,
 mais nous désirons être fidèles au Seigneur et à Sa Parole.
 Si la base de notre position sur le salut était fondée sur les œuvres
 et sur le comportement des gens et leur attitude,
 nous n'en serions que plus découragés et préoccupés.
 Mais lorsque nous gardons nos yeux fixés sur le Seigneur et gardons confiance en Lui seul
 et en Sa puissance, nous disons alors avec Pierre dans \ibibleverse{IP}(1:3-9)~:
 \og Béni soit Dieu, le Père de notre Seigneur Jésus-Christ, qui, selon sa grande miséricorde,
 nous a régénérés, pour une espérance vivante, par la résurrection de Jésus-Christ d'entre les morts,
 pour un héritage qui ne se peut ni corrompre, ni souiller, ni flétrir,
 lequel vous est réservé dans les cieux, à vous qui, par la puissance de Dieu,
 êtes gardés par la foi pour le salut prêt à être révélé dans les derniers temps!
 C'est là ce qui fait votre joie, quoique maintenant, puisqu'il le faut,
 vous soyez attristés pour un peu de temps par diverses épreuves, afin que l'épreuve de votre foi,
 plus précieuse que l'or périssable (qui cependant est éprouvé par le feu),
 ait pour résultat la louange, la gloire et l'honneur, lorsque Jésus-Christ apparaîtra,
 lui que vous aimez sans l'avoir vu, en qui vous croyez sans le voir encore,
 vous réjouissant d'une joie ineffable et glorieuse,
 parce que vous obtiendrez le salut de vos âmes pour prix de votre foi. \fg{}


\end{enumerate}

Il n'est pas facile de maintenir l'unité de l'Esprit parmi nous en ces matières.
 Il semble que la souveraineté de Dieu et la responsabilité humaine
 sont comme deux lignes parallèles qui ne se croiseront jamais dans nos esprits finis.
 Les voies de Dieu sont \og incompréhensibles \fg{} (\ibibleverse{Rm}(11:33))
 et la Bible nous met en garde de \og ne pas nous appuyer sur notre sagesse \fg{}
 (\ibibleverse{Pr}(3:5)).
 Dire ce que Dieu dit dans la Bible \ocadr ni plus, ni moins \fcadr{} n'est pas toujours facile,
 peut nous mettre mal à l'aise, ou peut paraître complètement incompréhensible.
 Mais l'Écriture Sainte nous dit que la sagesse d'en haut sera aimante
 et douce envers tous ceux qui recherchent l'unité des croyants,
 n'essayant pas de diviser et de séparer les uns des autres.
 Que Dieu nous aide tous à s'aimer, être bon, sensible, se pardonnant comme Jésus-Christ
 nous a pardonnés (\ibibleverse{Ep}(4:32))!
 Que dans les questions doctrinales difficiles, nous puissions nous comporter avec grâce
 et avec des cœurs remplis d'humilité, désirant avant tout Lui plaire,
 Lui qui nous a appelés à Le servir dans le corps de Christ.
 Discussion \ocadr \textsc{Oui} !
 Désaccords \ocadr \textsc{Oui} !
 Division \ocadr \textsc{Non} !

Jésus a dit~: \og Par leurs fruits vous les reconnaîtrez. \fg{}
 Dès qu'une position particulière sur les Écritures Saintes
 est source de conflit, de légalisme ou de division,
 je mets en doute la légitimité de cette position.
 Je cherche plutôt à tendre vers les choses qui vont me rendre plus aimant et bon,
 pardonnant plus et étant plus miséricordieux \fixme{mercy = miséricorde}.
 Je sais alors que je suis en train d'être transformé de plus
 en plus à l'image de mon Seigneur.
Si vous êtes arrivés à une intime conviction sur une question doctrinale particulière,
 merci de nous accorder le privilège de comprendre comment cela vous a aidé à devenir
 plus semblable à Christ dans votre nature, et ensuite nous jugerons
 si nous avons besoin d'aquérir la même conviction.
 Assurons-nous toujours de regarder le fruit de l'enseignement.
 \fixme{Contre-sens}

Recherchez les choses qui produisent la nature aimante de Jésus dans nos vies.
 Je préfèrerais avoir une mauvaise connaissance mais une attitude juste,
 plutôt qu'une connaissance juste et une mauvaise attitude.
 Dieu peut changer ma compréhension des choses en un instant,
 mais cela prend bien souvent toute une vie pour que mon attitude
 soit transformée.

Affectueusement,

\signature{Chuck Smith}


