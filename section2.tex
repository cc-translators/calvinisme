\section{Le Calvinisme}

De façon intéressante, Jean Calvin, le réformateur français, n'a pas élaboré
 ce que nous connaissons aujourd'hui sous le nom des cinq points du calvinisme.
 Ils ont été formés par le canon du Conseil de Dort (1618), et des déclarations
 complémentaires ont été développées plus tard à travers les nombreuses
 Confessions Réformées en ces matières. Le calvinisme est bien connu
 par les théologiens universitaires, prédicateurs, réformateurs,
 des hommes  remarquables tels que John Owen, George Whitefield,
 William Wilberforce, Abraham Kuyper, Charles Hodge, B.B Warfield,
 J.~Gresham Machen et Charles Haddon Spurgeon.

\begin{enumerate}

  \item  \textsc{Dépravation totale}

Les Calvinistes croyaient que l'homme est totalement \fixme{is in absolute}
 asservi à sa nature pécheresse et à Satan.
 Il est donc incapable de choisir le bien plutôt que le mal
 et de placer sa confiance en Jésus-Christ sans l'aide de Dieu.

  \item  \textsc{Élection inconditionnelle}

Les calvinistes croyaient à l'omniscience basée sur le plan et le but de Dieu~:
 Dieu choisit certains individus avant la fondation du monde pour qu'ils soient sauvés.
 Ce choix était uniquement motivé par sa propre volonté.

  \item  \textsc{La rédemption particulière ou l'expiation limitée}

les calvinistes croyaient que Jésus-Christ était mort pour ne sauver que les élus choisis par le Père pour l'éternité. Selon leur point de vue, Jésus est mort (pour les élus) uniquement pour ceux qui seront sauvés et que tous ceux pour qui Il n'est pas mort (les non-élus) seront perdus.

  \item  \textsc{L'appel efficace de l'esprit ou grâce irrésistible}

les Calvinistes croyaient que le Seigneur possède cette grâce irrésistible qui ne peut être entravée. Ils enseignaient qu'il y avait une grande distance entre le libre-arbitre de l'homme et le salut, les élus étant préparés (à devenir spirituellement vivants) par Dieu Lui-même bien avant qu'ils  placent leur foi en Jésus Christ en vue du salut.

  \item  \textsc{La persevérance des saints}

les calvinistes croyaient que le salut est entièrement le travail du Seigneur et que l'homme n'a absolument rien à voir dans la démarche. Ils pensaient que les saints doivent persévérer car Dieu finit toujours le travail qu'Il a commencé dans chaque croyant.

\end{enumerate}

