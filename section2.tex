%\chapter{Le calvinisme}

\digestpagebreak
\pocketpagebreak
\begin{digestpar}{\tolerance=2000}
\begin{pocketpar}{\tolerance=6000}
De façon intéressante, Jean Calvin, le réformateur français, n'a pas formulé
 ce que nous con\-naissons au\-jour\-d'hui sous le nom des cinq points du calvinisme.
 Ils ont été énoncés dans les Canons du Synode de Dordrecht (1618), et des déclarations
 complémentaires ont été développées plus tard à travers les nombreuses
 confessions réformées en ces matières. Le calvinisme est bien connu
 pour \fixme{Contre-sens: know for != known by}
 ses remarquables érudits, théologiens, prédicateurs et réformateurs,
 des hommes tels que John Owen, George Whitefield,
 William Wilberforce, Abraham Kuyper, Charles Hodge,
 B.~B~Warfield, J.~Gresham Machen et
 Charles Haddon Spurgeon.
\end{pocketpar}
\end{digestpar}

Ceux qui, au sein de la Réforme, ont répondu aux enseignements d'Arminius
 ont choisi le mot \Og \mbox{TULIP} \Fg{} comme acrostiche pour résumer
 leur réponse aux cinq points de l'arminianisme\frcolon

\nobreak

\newcommand*{\sectitle}[1]{%
   \normalsize(#1)}
\newcommand*{\secwithtitle}[2]{%
   \section*{#1\\\sectitle{#2}}}

\secwithtitle{La dépravation totale}{ T = Total Depravity}

Les calvinistes croyaient que l'hom\-me est totalement \fixme{is in absolute}
 asservi au péché \fixme{péché != nature pécheresse} et à Satan,
 et incapable d'exercer son libre arbitre 
 pour placer sa confiance en Jésus-Christ sans l'aide de Dieu.

\secwithtitle{L'élection inconditionnelle}{U = Unconditional Election}

Les calvinistes croyaient que la prescience
 est basée sur le plan et le but de Dieu, et que l'élection
 n'est pas basée sur la décision de l'hom\-me, mais sur le seul \Og libre arbitre \Fg{}
 du Créateur.

\secwithtitle{L'expiation limitée}{L = Limited Atonement}

\begin{pocketpar}{\tolerance=6000}
Les calvinistes croyaient que Jésus-Christ était mort pour sauver
 ceux qui Lui avaient été confiés par le Père de toute éternité.
 Selon leur point de vue, tous ceux pour qui Jésus est mort (les élus) seront sauvés,
 et tous ceux pour qui il n'est pas mort (les non-élus) seront \mbox{perdus}.
\end{pocketpar}

\secwithtitle{La grâce irrésistible}{I = Irresistible Grace}

\begin{pocketpar}{\tolerance=6000}
Les calvinistes croyaient que le Seigneur possède une grâce irrésistible
 qui ne peut être entravée. Ils enseignaient que le libre arbitre de l'hom\-me
 est si éloigné du salut, que les élus sont régénérés (amenés à la vie spirituellement)
 par Dieu avant
 même d'exprimer leur foi en Jésus-Christ en vue du salut.
 Si une per\-son\-ne totalement pervertie n'était pas amenée à la vie par l'Esprit-Saint,
 un tel appel de Dieu serait impossible.
\end{pocketpar}

\secwithtitle{La persevérance des saints}{P = Perseverance of the Saints}

Les calvinistes croyaient que le
 salut est entièrement le travail du Seigneur
 et que l'homme n'a absolument rien à voir dans la démarche.
 Les saints vont persévérer car Dieu va s'as\-surer de finir le travail
 qu'Il a commencé.


%\closechapter
