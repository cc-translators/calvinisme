\chapter{Le calvinisme}

\lettrine{D}{e façon intéressante,} Jean Calvin, le réformateur français, n'a pas élaboré
 ce que nous connaissons aujourd'hui sous le nom des Cinq Points du calvinisme.
 Ils ont été énoncés par le canon du Conseil de Dort (1618), et des déclarations
 complémentaires ont été développées plus tard à travers les nombreuses
 Confessions Réformées en ces matières. Le calvinisme est bien connu
 pour \fixme{Contre-sens: know for != known by}
 ses remarquables érudits, théologiens, prédicateurs et réformateurs,
 des hommes tels que John Owen, George Whitefield,
 William Wilberforce, Abraham Kuyper, Charles Hodge, \pocketlinebreak
 B.B Warfield, J.~Gresham Machen et \pocketlinebreak
 Charles Haddon Spurgeon.

Ceux qui, au sein de la Réforme, ont répondu aux enseignements d'Arminius
 ont choisi le mot \og TULIP \fg{} comme acrostiche pour résumer
 leur réponse aux Cinq Points de l'arminianisme~:

\begin{enumerate}

  \item  \textsc{La dépravation totale\\\nobreak (\og T \fg{} = Total Depravity)}

\nobreak
Les Calvinistes croyaient que l'hom\-me est totalement \fixme{is in absolute}
 asservi au péché \fixme{péché != nature pécheresse} et à Satan,
 et incapable d'exercer son libre arbitre 
 pour placer sa confiance en Jésus-Christ sans l'aide de Dieu.
 \fixme{Le terme « libre arbitre » est important en réponse au point 1 de l'arminianisme}

  \item  \textsc{L'élection inconditionnelle\\\nobreak (\og U \fg{} = Unconditional Election)}

\nobreak
Les calvinistes croyaient que la prescience \fixme{prescience != omniscience}
 est basée sur le plan et le but de Dieu, et que l'élection
 n'est pas basée sur la décision de l'homme, mais sur le seul \og libre arbitre \fg{}
 du Créateur.

\pocketpagebreak
  \item  \textsc{L'expiation limitée\\\nobreak (\og L \fg{} = Limited Atonement)}
  \fixme{Il vaut mieux choisir le terme choisi par l'auteur}

\nobreak
Les calvinistes croyaient que Jésus-Christ était mort pour sauver
 ceux qui Lui avaient été confiés par le Père de toute éternité.
 \fixme{Pour l'éternité = éternité future ; de toute éternité = éternité passée}
 Selon leur point de vue, tous ceux pour qui Jésus est mort (les élus) seront sauvés,
 et tous ceux pour qui il n'est pas mort (les non-élus) seront perdus.

  \item  \textsc{La grâce irrésistible\\\nobreak (\og I \fg{} = Irresistible Grace)}
  \fixme{Il vaut mieux choisir le terme choisi par l'auteur}

\nobreak
\begin{specialpar}{\tolerance=3000}
Les Calvinistes croyaient que le Seigneur possède une grâce irrésistible
 qui ne peut être entravée. Ils enseignaient que le libre arbitre de l'hom\-me
 est si éloigné du salut, que les élus sont régénérés (amenés à la vie spirituellement)
 par Dieu avant \pocketlinebreak
 même d'exprimer leur foi en Jésus-Christ en vue du salut.
 Si une per\-son\-ne totalement pervertie n'était pas amenée à la vie par l'Esprit-Saint,
 un tel appel de Dieu serait impossible.
\end{specialpar}

  \item  \textsc{La persevérance des saints\\\nobreak (\og P \fg{} = Perseverance of the Saints)}

\nobreak
\begin{specialpar}{\tolerance=1000}
Les calvinistes croyaient que le \pocketlinebreak
 salut est entièrement le travail du Seigneur
 et que l'homme n'a absolument rien à voir dans la démarche.
 Les saints vont persévérer car Dieu va s'assurer de finir le travail
 qu'Il a commencé.
\end{specialpar}

\end{enumerate}


\closechapter
