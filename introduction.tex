\addchap{Introduction}

\begin{specialpar}{\tolerance=2000}
\lettrine{Q}{ue signifie} faire partie du nombre \pocketlinebreak
 croissant d'églises Calvary Chapel?
\fixme{On parle en général des églises Calvary Chapel}
 Nous avons différentes manières de nous distinguer des autres églises évangéliques.
 Nous pourrions mettre l'accent sur notre partage de l'enseignement systématique \pocketlinebreak
 de la Bible
 ou sur l'amour qui dépasse tou\-tes les frontières culturelles ou ethniques.
Calvary Chapel est reconnu pour se focaliser sur la louange,
 notamment par un
 accompagnement de musique con\-tem\-po\-rai\-ne fidèle à la Parole de Dieu,
 et en vue d'attirer Son Peuple à louer son Seigneur.
 Sans exception, les églises Calvary Chapel se sont prononcées très fermement
 en faveur d'une vision pré-tribulationniste et pré-millénariste du retour de Jésus-Christ.
\fixme{Contre-sens~: les pré-tribulationnistes croient que Jésus revient enlever l'église avant la tribulation, les pré-millénaristes croient que le millénium a lieu après l'enlèvement de l'église}
Nous avons également exprimé un amour et un soutien constants pour la nation
 d'Israël,
 dans son droit à une patrie historique  et dans son devoir de reconnaître le Messie.
 Mais le plus important est que Calvary Chapel soit connu pour avoir su trouver
 le juste milieu entre les opinions extrêmes sur des questions théologiques
 controversées qui bien souvent ont causé la division plutôt que l'unité
 à travers le corps du Christ.
\end{specialpar}

Calvary Chapel ne souhaite ni diviser, ni être dogmatique sur les domaines où les partisans
 ou les enseignants de la Bible n'ont pas réussi à s'entendre.
 Cependant, il est important d'exposer aussi clairement que possible la base doctrinale
 de notre communauté et de rester unis, en particulier dans le domaine de l'autorité pastorale
 et de l'enseignement. De même que nous accueillons volontiers les croyants qui
 n'ont pas
 le même point de vue que notre communauté, nous les encourageons à 
 faire preuve de compréhension
 à l'égard de la doctrine dans une certaine mesure, et ceci, afin de garder un esprit uni
 parmi les pasteurs enseignant les vérités de la Parole du Seigneur.

Les églises Calvary Chapel essaient \pocketlinebreak
 d'éviter les conclusions hâtives, les terminologies,
 ou les disputes
 \fixme{Contre-sens~: argument = dispute, querelle}
 sur des sujets qui ne sont pas explicites dans la Bible.
 Il n'y a sans aucun doute pas de polémique plus passionnée que le débat sans fin
 entre les Calvinistes et les Arminianistes. Au milieu de cette discussion houleuse,
 il est très facile de méconnaître ou de négliger les principes évidents de la Bible
 ou de croire que nous avons la capacité de comprendre entièrement les voies du Seigneur
 (\ibibleverse{Rm}(11:33-36)).
 Comme il est tragique de vouloir absolument avoir raison plutôt que de se montrer aimant.
 Quand nous débattons sur le ministère du Saint-Esprit, il est facile de ne pas être d'accord
 sur le sens des mots, tels que le \og baptême \fg{}, ou \og être rempli \fg{} 
 et de passer à côté de la bénédiction et de la puissance de l'Esprit de Dieu dans nos vies.
 La manière dont nous conduisons les débats ou dont nous nous exprimons pourra parfois
 aussi bien \og éteindre \fg{}  qu'\og attrister\fg{} 
 l'Esprit béni qui demeure en chaque croyant.
 Au milieu de tous nos différends au sujet des dons spirituels,
 \fixme{Contre-sens~: over the gifts = au sujet des dons}
 nous pouvons laisser échapper la recommandation biblique d'aimer;
 évidemment le plus grand de tous les dons (\ibibleverse{ICo}(12:31)-(14:1)).
 Notre désir est de rassembler les croyants dans l'amour et dans l'unité du Saint-Esprit.
 Nous portons notre attention sur notre Dieu extraordinaire, et non sur
 nous-mêmes.
 Nous avons à cœur de glorifier notre Seigneur dans tout ce que nous disons ou faisons.

Il n'y a sans doute aucune question plus importante que celle de la doctrine du salut,
 reflétée probablement le plus dans le débat entre les disciples de Jean Calvin (1509--1564)
 et ceux de Jacob Hermann \pocketlinebreak (1560--1609), plus connu sous son nom latin, Arminius.
 Depuis la réforme protestante au 16\up{e} siècle, les églises chrétiennes
 et leurs dirigeants ne se sont pas entendus sur les questions telles que la dépravation,
 la souveraineté de Dieu, la responsabilité humaine, l'élection, la sécurité éternelle,
 et la nature et l'étendue de l'expiation de Jésus-Christ.

\begin{specialpar}{\tolerance=3000}
Bien que formé dans la tradition de la Réforme, Arminius a eu de sérieux doutes
 à propos de \og la grâce souveraine \fg{}  telle qu'enseignée
 par les disciples de Jean Calvin. Pasteur de la congrégation réformée à Amsterdam (1588),
 il commença de se poser des questions sur les conclusions du calvinisme
 durant ses 15~années de ministère en ce lieu. Abandonnant le pastorat,
 il devint professeur de théologie à l'Université de Leyden.
 Ses enseignements sur l'élection et la prédestination menèrent à une violente
 et tragique polémique. Après sa mort en 1609, ses disciples développèrent la Protestation
 de 1610 qui se résume dans les \og cinq points de l'arminianisme \fg{}.
 Ce document était une protestation contre les doctrines des calvinistes,
 et avait été soumis à l'État de Hollande. En 1618, la Con\-vention Nationale
 de l'Église était réunie à Dort pour examiner les enseignements d'Arminius
 à la lumière des Écritures. Après 154~sessions, étendues sur
 7~mois,
 les Cinq Points de l'arminianisme furent déclarés hérétiques.
 Après la
 Con\-ven\-tion, plusieurs des disciples d'Ar\-mi\-nius, comme Hugo Grotius,
 furent emprisonnés ou exilés. Quand John Wesley reprit certains des enseignements
 de l'armi\-nia\-nisme, le mouvement commença de croître, et cela affecta la tradition méthodiste
 ainsi que les convictions de la majorité des églises pentecôtistes et charismatiques.
\end{specialpar}


\closeintro

