%\addchap{Introduction}
\addchap{Calvinisme,\\ Arminianisme\\ \& Parole de Dieu}

\begin{pocketpar}{\tolerance=2500}
\begin{digestpar}{\tolerance=3000}
\lettrine{Q}{ue signifie} faire partie du nombre \pocketlinebreak
 croissant d'églises Calvary Chapel ?
\fixme{On parle en général des églises Calvary Chapel}
 Certains aspects nous démarquent des autres églises évangéliques.
 Nous pourrions mettre en avant notre engagement commun à enseigner la Bible
 de façon systématique ou l'accent mis sur l'amour qui dépasse
 toutes les frontières culturelles ou ethniques.
Les églises Calvary Chapel sont également connues pour l'importance centrale
 qu'elles accordent à la louange, par l'intermédiaire de musique con\-tem\-po\-rai\-ne
 fidèle à la Parole de Dieu et au désir de Son peuple de Le louer.
 Sans exception, les églises Calvary Chapel se prononcent très fermement
 en faveur d'une vision pré-tribulationniste et pré-millénariste du retour de Jésus-Christ.
\fixme{Contre-sens~: les pré-tribulationnistes croient que Jésus revient enlever l'église avant la tribulation, les pré-millénaristes croient que le millénium a lieu après l'enlèvement de l'église}
Nous avons également exprimé un amour et un soutien constants pour la nation
 d'Israël,
 dans son droit à une patrie historique et dans son besoin du Messie.
 Mais le plus important, c'est que Calvary Chapel soit connu pour avoir su trouver
 le juste milieu entre les opinions extrêmes sur des questions théologiques
 controversées qui bien souvent ont causé la division plutôt que l'unité
 au sein du corps du Christ.
\end{digestpar}
\end{pocketpar}

\begin{pocketpar}{\tolerance=2000}
Les églises Calvary Chapel ne désirent pas diviser, ni être dogmatiques
 dans les domaines où les partisans et les enseignants de la Bible
 n'ont pas réussi à s'entendre.
 Cependant, il est important d'affirmer aussi clairement que possible
 notre base doctrinale commune et notre unité, en particulier dans le domaine
 de l'autorité pastorale et de l'enseignement. Bien que nous accueillions
 volontiers dans notre mouvement les croyants qui ne partagent pas notre point de vue,
 nous encourageons en revanche une unité de vision doctrinale de la part des pasteurs
 qui ont la responsabilité de nous enseigner les vérités de la Parole de Dieu.
\end{pocketpar}

\begin{pocketpar}{\tolerance=3000}
\begin{digestpar}{\tolerance=1000}
Les églises Calvary Chapel essaient \pocketlinebreak
 d'éviter les prises de position, les terminologies et les querelles
 \fixme{Contre-sens~: argument = dispute, querelle}
 sur des sujets qui ne sont pas clairement présentés dans la Bible.
 Il n'y a pas de sujet de controverse où cette approche a plus d'importance
 que dans le débat sans cesse attisé entre calvinistes et arminiens.
 Au milieu de cette discussion houleuse, il est facile de méconnaître ou de négliger
 les principes simples de la Bible ou de croire que nous avons la capacité
 de comprendre entièrement les voies du Seigneur
 (\ibibleverse{Rm}(11:33-36)).
 Mais quelle tragédie lorsque nous nous inquiétons plus d'avoir \og raison \fg{}
 que de faire preuve d'amour !
 Quand nous débattons sur le ministère du Saint-Esprit, il est facile de ne pas être d'accord
 sur des termes tels que \og baptême \fg{} ou \og être rempli \fg{} 
 et de passer à côté de la bénédiction et de la puissance de l'Esprit de Dieu dans nos vies.
 La manière dont nous conduisons les débats ou dont nous nous exprimons pourra parfois
 \og éteindre \fg{} ou encore \og attrister\fg{} l'Esprit béni qui demeure en chaque croyant.
 Au milieu de tous nos différends au sujet des dons spirituels,
 \fixme{Contre-sens~: over the gifts = au sujet des dons}
 nous pouvons passer à côté de l'exhortation biblique d'aimer,
 qui est clairement plus grande que tous les dons (\ibibleverse{ICo}(12:31)-(14:1)).
 Notre désir est de rassembler les croyants dans l'amour et dans l'unité du Saint-Esprit.
 Nous fixons notre attention sur notre Dieu grandiose, et non sur
 nous-mêmes.
 Nous sommes dévoués à glorifier notre Seigneur dans tout ce que nous disons ou faisons.
\end{digestpar}
\end{pocketpar}

\begin{pocketpar}{\tolerance=3500}
Il n'y a probablement pas de question plus importante et qui puisse générer autant
 de divisions que celle de la doctrine du salut, qui est au cœur du débat
 entre les disciples de Jean Calvin (1509--1564) et ceux de Jacob Hermann (1560--1609),
 plus connu sous son nom latin, Arminius.
 Depuis la réforme protestante au 16\up{e} siècle, les églises chrétiennes
 et leurs dirigeants ont été en désaccord sur des questions telles que la dépravation,
 la souveraineté de Dieu, la responsabilité humaine, l'élection,
 la prédestination, la sécurité éternelle,
 et la nature et l'étendue de l'expiation de Jésus-Christ.
\end{pocketpar}

\begin{pocketpar}{\tolerance=3500}
\begin{digestpar}{\tolerance=1000}
Bien que formé dans la tradition de la Réforme, Arminius a eu de sérieux doutes
 sur la doctrine de la \og grâce souveraine \fg{} telle que les disciples de Jean Calvin
 l'enseignaient. Il était pasteur de la congrégation réformée d'Amsterdam (1588),
 mais au cours des quinze années de ce ministère, il commença à remettre en question
 les conclusions du calvinisme.
 Il abandonna le pastorat et devint professeur de théologie à l'Université de Leyde.
 Ce sont ses enseignements sur l'élection et la prédestination qui menèrent à une violente
 et tragique polémique. Après sa mort en 1609, ses disciples développèrent les Remontrances
 de 1610 qui définissaient les \og cinq points de l'arminianisme \fg{}.
 Ce document était une protestation contre les doctrines des calvinistes,
 et fut soumis à l'État de Hollande. En 1618, un Synode National de l'Église
 fut convoqué à Dordrecht pour examiner les enseignements d'Arminius
 à la lumière des Écritures. Après 154~sessions, étendues sur
 sept mois, les cinq points de l'arminianisme furent déclarés hérétiques.
 Après le synode, beaucoup des disciples d'Arminius, comme Hugo Grotius,
 furent emprisonnés ou exilés. Quand John Wesley reprit certains des enseignements
 de l'armi\-nia\-nisme, le mouvement commença de croître, et cela affecta
 la tradition méthodiste ainsi que les convictions de la majorité
 des églises pentecôtistes et charismatiques.
\end{digestpar}
\end{pocketpar}


%\closeintro

